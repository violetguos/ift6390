\documentclass[letterpaper, 12pt]{article} 

\usepackage[margin=0.75in]{geometry}
\usepackage{tabularx}
\usepackage{amsmath,amsthm,amssymb,enumerate,graphicx,leading}
\usepackage{amsfonts, physics, bbold, bm, fancyvrb}
\usepackage{listings}
\usepackage{color}

\usepackage{fullpage} %Package to use full page
\usepackage{relsize}
\usepackage{palatino} % this has a bigger font

% For nice pseudocode
\definecolor{dkgreen}{rgb}{0,0.6,0}
\definecolor{gray}{rgb}{0.5,0.5,0.5}
\definecolor{mauve}{rgb}{0.58,0,0.82}

\lstset{frame=tb,
  language=Python,
  aboveskip=3mm,
  belowskip=3mm,
  showstringspaces=false,
  columns=flexible,
  basicstyle={\small\ttfamily},
  numbers=none,
  numberstyle=\tiny\color{gray},
  keywordstyle=\color{blue},
  commentstyle=\color{dkgreen},
  stringstyle=\color{mauve},
  breaklines=true,
  breakatwhitespace=true,
  tabsize=3
}

% For problem/solution setup
\usepackage[justification=centering]{caption}
\leading{18pt}
\newtheorem{problem}{Problem} 
\theoremstyle{definition} 
\newtheorem*{solution}{Solution} 

\begin{document} 

% For floor and ceiling symbols
\def\lc{\left\lceil}   
\def\rc{\right\rceil}
\def\lf{\left\lfloor}
\def\rf{\right\rfloor}

%Good resources for looking up how to do stuff:
%Binary operators: http://www.access2science.com/latex/Binary.html
%General help: http://en.wikibooks.org/wiki/LaTeX/Mathematics
%Or just google stuff
 
\title{IFT 6390 Homework 2}
\author{Arlie Coles (20121051) and Yue (Violet) Guo (20120727)}
\maketitle

%%% Problem 1 %%%
\section{Linear and non-linear regularized regression}

\begin{problem}
    Linear regression.
\end{problem}
\begin{solution}\
    \begin{enumerate}
        \item 
        \item
        \item
        \item
        \item
    \end{enumerate}
\end{solution}\

\begin{problem}
    Ridge regression.
\end{problem}
\begin{solution}\
    \begin{enumerate}
        \item 
        \item
        \item
        \item
    \end{enumerate}
\end{solution}\

\begin{problem}
    Regression with a fixed non-linear preprocessing.
\end{problem}
\begin{solution}\
    \begin{enumerate}
        \item 
        \item
        \item
        \item
    \end{enumerate}
\end{solution}

\section{Practical part}

Please see the attached iPython notebook for the code and writup for the practical part. \textit{Note to TAs: our notebook runs on Python 3.}

\end{document}